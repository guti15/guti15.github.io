%%%%%%%%%%%%%%%%%%%%%%%%%%%%%%%%%%%%%%%%%
% Medium Length Professional CV
% LaTeX Template
% Version 2.0 (8/5/13)
%
% This template has been downloaded from:
% http://www.LaTeXTemplates.com
%
% Original author:
% Trey Hunner (http://www.treyhunner.com/)
%
% Important note:
% This template requires the resume.cls file to be in the same directory as the
% .tex file. The resume.cls file provides the resume style used for structuring the
% document.
%
%%%%%%%%%%%%%%%%%%%%%%%%%%%%%%%%%%%%%%%%%

%----------------------------------------------------------------------------------------
%	PACKAGES AND OTHER DOCUMENT CONFIGURATIONS
%----------------------------------------------------------------------------------------

\documentclass{resume} % Use the custom resume.cls style

\usepackage[left=0.75in,top=0.6in,right=0.75in,bottom=0.6in]{geometry} % Document margins
\usepackage{pdfsync}
\usepackage{hyperref}
\hypersetup{%
  colorlinks= true,% hyperlinks will be coloured
  linkcolor= green,% hyperlink text will be green
  urlcolor = blue, % makes text of Description appear blue
  }

\name{Robert Gutierrez} % Your nameo
\address{20 Chester Place  \\ Lynn, Ma 01904} % Your secondary addess (optional)
\address{(781)~$\cdot$~426~$\cdot$~1370 \\ \href{mailto:guti310@gmail.com}{guti310@gmail.com}} % Your phone number and email

\begin{document}

%----------------------------------------------------------------------------------------
%	EDUCATION SECTION
%----------------------------------------------------------------------------------------

\begin{rSection}{Education}

{\bf Wheaton College, Ma} \hfill {\em May 2015} \\ 
Bachelor of Arts - Computer Science

% {\b
% Lawrence Academy, Ma} \hfill {\em May 2011} \\ 
% High School                     

\end{rSection}



%----------------------------------------------------------------------------------------
%	WORK EXPERIENCE SECTION
%----------------------------------------------------------------------------------------
% [ Rubik's Cube - LEAP MOTION ] 

\begin{rSection}{Experience}

\begin{rSubsection}{Leap Motion Rubik's Cube}{August 2013 - December 2013}{Developer/Designer}{Norton, MA}
\item With a group of 4, built a Rubik's Cube application. Using \href{https://www.leapmotion.com/}{Leap Motion Controller} when it was recently released we decided to make a Rubik's Cube application whiche made it possible to use hand gestures to manipulate the cube. 

\item Tools used were C++  and open GL. Additionaly made the application cross platform compatible between Windows and Mac OS.  

\end{rSubsection}

%------------------------------------------------
% Natural Language

\begin{rSubsection}{Natural Language Processing}{January 2014 - May 2014}{Student}{Norton, MA}
\item Worked with Natural Language Toolkit in Python where we learned the basics of languguae processing.  Involved breaking up sentence, and working with speach recognition. Additionally got to work AT\&T Speach Recognition API. 



%------------------------------------------------
% Personal Webpage
\begin{rSubsection}{Personal Website}{November 2013 - September 2014}{Editor}{Norton, MA}
\item  Maintained a personal website, began to familirize myself with front end development. Additonally required to use github pages and fimilirize myself more with version control. \href{http://guti15.github.io/}{Website}


\end{rSubsection}

%------------------------------------------------
% [ Emacs DOCUMENTATION  ]

\begin{rSubsection}{Emacs Documentation ES}{January 2014 - April 2014}{Editor}{Norton, MA}
\item As an Emacs enthusiast began a documentation in Spanish, that allowed native Spanish Speakers to use Emacs.  This was created due to limited resources in Spanish. 
\item Webpage documented links where you could download tools to get started, displayed basic movement that is crucial to getting started.  Additionally created a small python code that implemented further movement inside the emacs editior. \href{http://guti15.github.io/Emacs_Documentation_es/Intro.html}{Documentation}

\end{rSubsection}


%------------------------------------------------
%NFC Application 
\begin{rSubsection}{NFC Application}{April 2015 - Present}{Android Developer}{Norton, MA}
\item Developing an Android application that uses Near Field Communication in order to exchage contact information, including phone number and social media information.

\item tools being used Android SDK and Java. 



\end{rSubsection}

\end{rSection}

%----------------------------------------------------------------------------------------
%	TECHNICAL STRENGTHS SECTION
%----------------------------------------------------------------------------------------

\begin{rSection}{Technical Strengths}

\begin{tabular}{ @{} >{\bfseries}l @{\hspace{6ex}} l }
Computer Languages\\(sorted by fimilarity) & C++, Python, Java, Javascript , Bash, LISP \\
Other Programing Language and Tools & Emacs, Git/Github, \LaTeX, HTML5, CSS3,   \\
Operating Systems & Mac OS, Windows, Ubuntu 12.04 \\

\end{tabular}

\end{rSection}

%----------------------------------------------------------------------------------------
%	EXAMPLE SECTION
%----------------------------------------------------------------------------------------

%\begin{rSection}{Section Name}

%Section content\ldots

%\end{rSection}

%----------------------------------------------------------------------------------------

\end{document}
